\begin{abstract}

  In the past few decades, researchers have been studying the motion pattern of bird flocks, learning from biomimetics and imitating their behaviors with small scale mobile robots due to its potential applications in those environments where self-organizing, healing and configuring capabilities are desired. Various control laws have been developed for mass particles and verified with simulation results, however, few of them have been implemented on real unmanned aerial vehicles (UAVs). On the flip side, existing UAV flockings rely heavily on centralized control or measurement, which fail to percept and determine their individual movement like natural birds. This leaves us an empty space worthwhile to discover: the realization of UAV flockings with distributed control algorithms.

  In this thesis, we first review flocking related consensus problems, propose a second-order multi-agent system and prove its stability. Second, we present the design of our flocking system: one manually controlled leader UAV and one autonomous driving follower UAV with 1.12 kg total weight, 25 cm diagonal length and the monocular camera as the only on-board sensor. Last but not least, simulations, comparison with the traditional tracking algorithm and experiments in both indoor and outdoor GPS-denied environments are demonstrated. It is shown that our solution converges as fast as C-S model, enjoys shorter relative distance than C-D model and reacts faster than tracking algorithm.

\end{abstract}
