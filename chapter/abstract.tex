\begin{abstract}

  Collective behaviors are meaningful to natural animals, which could be frequently observed during group hunting, entertainment and traveling. In bird flocks, it is found that individual birds tend to form into a shape of letter V or a straight line to save body strength during long distance movement. In the past few decades, researchers have been studying the motion pattern of bird flocks, learning from biomimetics and imitating their behaviors with small scale mobile robots. Various control models have been developed for mass particles and verified with simulation results, however, few of them have been implemented on real aerial vehicles.

  Existing drone swarms or flocks either use central computer for relative information processing and publishing commands through wireless modules or rely heavily on motion capture system to obtain absolute global positions. Natural birds, however, rely less on internal communication, percept and determine their movement individually, which leaves us a challenging problem to worth to solve: to realize the flocking models with aerial robots without depending external sensors or centralized control.

  In this thesis, we have proposed a flocking model describing the control laws for individual agents in the flock in continuous time, proved its convergence and stability and realized it with our flocking system of two quadrotors. Simulation and experiments analysis in both indoor and outdoor GPS-denied environments are presented.

\end{abstract}
