\chapter{Conclusion}\label{conclusion}

In this thesis, a flocking system consisting two UAVs with our control law is presented and the detailed hardware and software architectures are introduced. The leader UAV is manually controlled and the follower UAV could be switched to autonomous mode and begins flocking with the leader. The monocular camera is used as the only on-board sensor for both follower UAV's state estimation and leader UAV's pose recognition. When in flocking, the follower UAV first percepts the surrounding environment to estimate self-state, then recognizes leader UAV's pose and calculates desired acceleration using our proposed model and third executes the desired input until next image is captured and processed.

Simulations, real world experiments and the comparison of the results from formation algorithm with our proposed method have been conducted and analysed. We have shown that our flocking system has met the three flocking criteria without relying on any external perception system or centralized control panel, achieved fast convergence rate and kept bounded relative distance with neighboring agents to avoid collision, given fine weather conditions.

Our future work will focus on the flocking of more than two quadrotors in GPS-denied environment, including extending our flocking model from fixed topology to dynamic topology, extending our flocking model form homogeneous to heterogeneous, introducing multiple fisheye cameras or an omnidirectional camera for perception and implementing ultra wide band (UWB) sensor for relative displacement measurement and internal communication.
