\chapter{Conclusion}\label{conclusion}

In this thesis, a flocking system consisting two quadrotors based on mathematical control model is presented and the detailed hardware and software platforms are introduced. The leader UAV is manually controlled, the follower UAV could be switched to autonomous mode and stay flocking with the leader. The monocular camera is used as the only on-board sensor for both follower UAV state estimation and leader UAV pose recognition. When the flocking begins, the follower UAV first percepts the surrounding environment to estimate self-state, then recognize leader UAV's pose and calculate desired acceleration using our proposed model and third execute the desired input until next image is captured and processed.

Simulations, real world experiments and the comparison of the results from tracking algorithm with our proposed method have been conducted and analysed. We have shown that our flocking system has met the three flocking criteria without relying on any external perception system or central control panel, achieved relatively faster convergence rate and kept bounded relative distance with neighboring agents to avoid collision, given fine weather conditions.

Our future work will focus on the flocking of more than two quadrotors in GPS-denied environment, including introducing multiple fisheye cameras or an omnidirectional camera for perception and the study of ultra wide band (UWB) method for relative distance measurement and internal communication.
